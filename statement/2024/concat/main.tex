\def\probtitle{온데간데없을뿐더러}
\def\probno{C} % 문제 번호

\begin{problem}{\probno{}. \probtitle{}}

`온데간데없을뿐더러’는 어떻게 띄어 써야 할까? 형용사 `온데간데없다’와 어미 `-을뿐더러’가 결합하였기 때문에 띄어쓰기 없이 `온데간데없을뿐더러’라고 쓰는 것이 올바른 표현이다.

각각 $N$개의 양의 정수로 구성된 두 배열 $A$, $B$가 주어진다. $A$에 있는 수를 모두 순서대로 공백 없이 이어서 썼을 때 얻게 되는 수를 $X$, $B$에 있는 수를 같은 방식으로 이어서 썼을 때 얻게 되는 수를 $Y$라고 했을 때, $X$와 $Y$ 중 더 작은 값을 구하는 프로그램을 작성하시오.

\InputFile

첫째 줄에 각 배열의 원소 개수 $N$이 주어진다.

둘째 줄에 배열 $A$의 원소 $A_1, A_2, \cdots, A_N$이 차례대로 공백으로 구분되어 주어진다.

셋째 줄에 배열 $B$의 원소 $B_1, B_2, \cdots, B_N$이 차례대로 공백으로 구분되어 주어진다.

\OutputFile

$X$와 $Y$ 중 더 작은 값을 출력한다. $X$와 $Y$가 같은 경우 그 값을 출력한다.

\Constraints

\begin{itemize}[topsep=0pt,noitemsep]
    \item $1 \le N \le 9$
    \item $1 \le A_i \le 99 (1 \le i \le N)$
    \item $1 \le B_j \le 99 (1 \le j \le N)$
    \item 입력으로 주어지는 수는 모두 정수이다.
\end{itemize}

\Example

\begin{example}
    \exmpfile{./example/01.in.txt}{./example/01.out.txt}%
    \exmpfile{./example/02.in.txt}{./example/02.out.txt}%
\end{example}

% \newpage

\Notes
$X$와 $Y$는 $2^{31} - 1$보다 클 수 있으므로 C/C++에서는 long long 타입, Java에서는 long 타입을 이용해 저장하는 것을 권장한다.

\end{problem}