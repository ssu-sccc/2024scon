\def\probno{B}
\def\probtitle{팀명 정하기 2}

\section{\probno{}. \probtitle{}}

\begin{frame} % No title at first slide
    \sectiontitle{\probno{}}{\probtitle{}}
    \sectionmeta{
        \texttt{string, implementation}\\
        출제진 의도 -- \textbf{\color{acbronze}Easy}
    }
    \begin{itemize}
        \item 처음 푼 팀: \textbf{떨거지들}, 6분
        \item 처음 푼 팀(Open Contest): \textbf{riroan}, 2분
        \item 출제자: 나정휘
    \end{itemize}
\end{frame}

\begin{frame}{\probno{}. \probtitle{}}
    \begin{itemize}
        \item 팀명 $S$에 등장하는 알파벳 소문자, 대문자, 숫자의 개수를 각각 $L, U, D$라고 합시다.
        \item 아래 세 가지 조건을 모두 만족하는 팀명을 찾아서 출력하면 됩니다.
        \begin{itemize}
            \item 대문자가 소문자보다 더 많이 등장하면 안 된다. ($U \le L$)
            \item 팀명은 10글자 이하로 지어야 한다. ($|S| \le 10$)
            \item 숫자가 아닌 글자가 하나 이상 포함되어 있어야 한다. ($D < |S|$)
        \end{itemize}
    \end{itemize}
\end{frame}
