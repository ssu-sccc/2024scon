\def\probno{J}
\def\probtitle{Traveling SCCC President 2}

\section{\probno{}. \probtitle{}}

\begin{frame} % No title at first slide
    \sectiontitle{\probno{}}{\probtitle{}}
    \sectionmeta{
        \texttt{graph\_traversal, bitmask}\\
        출제진 의도 -- \textbf{\color{acgold}Hard}
    }
    \begin{itemize}
        \item 처음 푼 팀: \textbf{N/A}, N/A분
        \item 처음 푼 팀(Open Contest): \textbf{dkim110807}, 18분
        \item 출제자: 오주원
    \end{itemize}
\end{frame}

\begin{frame}{\probno{}. \probtitle{}}
    \begin{itemize}
        \item $2^{60}-1$ 시간을 소요하면 모든 간선을 사용할 수 있으므로 항상 $N$번 정점에 도달할 수 있습니다.
        \item 소요 시간 $X$가 주어졌을 때 $N$번 정점에 도달 가능한지 판별할 수 있을까요?
    \end{itemize}
\end{frame}

\begin{frame}{\probno{}. \probtitle{}}
    \begin{itemize}
        \item 전체 소요 시간이 $X$이면 $w_i \wedge X = w_i$인 간선을 사용할 수 있습니다.
        \item 따라서 위 조건을 만족하는 간선만 사용하는 $1 \rightarrow N$ 경로가 존재하는지 확인하면 됩니다.
        \item 두 정점의 연결성 판별은 DFS, BFS 등을 이용해 $O(N+M)$ 시간에 해결할 수 있습니다.
    \end{itemize}
\end{frame}

\begin{frame}{\probno{}. \probtitle{}}
    \begin{itemize}
        \item 소요 시간을 최소화해야 하므로 되도록 높은 비트를 꺼진 상태로 만들어야 합니다.
        \item 따라서 $2^{59}$부터 $2^0$까지 하나씩 $k$번째 비트를 꺼보면서 문제의 정답이 될 수 있다면 유지, 정답이 될 수 없다면 다시 켜는 것으로 문제를 해결할 수 있습니다.
        \item 전체 시간 복잡도는 $O((N+M) \log W)$입니다.
    \end{itemize}
\end{frame}
