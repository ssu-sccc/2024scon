\def\probtitle{과민성 대장 증후군}
\def\probno{A} % 문제 번호

\begin{problem}{\probno{}. \probtitle{}}

상원이는 과민성 대장 증후군을 앓고 있다. 과민성 대장 증후군의 원인은 스트레스!

상원이의 $N$일 동안의 스트레스 변화량 $A_1, \cdots, A_N$이 주어진다. $A_i \ge 0$ 이면 $i$번째 날에 $A_i$ 만큼 스트레스가 쌓이고, $A_i < 0$ 이면 $i$번째 날에 $-A_i$ 만큼 스트레스가 해소된다. 단, 변화를 관찰하기 시작한 시점의 스트레스 양은 $0$이며, 누적된 스트레스 양보다 해소하는 스트레스 양이 더 많을 경우 스트레스는 $0$이 될 때까지만 감소한다.

상원이는 스트레스가 $M$ 이상 쌓인 날에 복통을 겪게 될 때, 상원이가 며칠 동안 복통에 시달리게 되는지 알아보자.

\InputFile

첫째 줄에 스트레스 변화를 관찰한 일수 $N$과 복통을 겪게 되는 스트레스의 양 $M$이 주어진다.

둘째 줄에 스트레스 변화량 $A_1, A_2, \cdots, A_N$이 공백으로 구분되어 주어진다.

\OutputFile

상원이가 복통을 겪게 되는 일수를 출력한다.

\Constraints

\begin{itemize}[topsep=0pt,noitemsep]
    \item $1 \leq N \leq 10^5$
    \item $1 \leq M \leq 10^9$
    \item $-10^4 \leq A_i \leq 10^4 (1 \le i \le N)$
    \item 입력으로 주어지는 수는 모두 정수이다.
\end{itemize}

\Example

\begin{example}
    \exmpfile{./example/01.in.txt}{./example/01.out.txt}%
    \exmpfile{./example/02.in.txt}{./example/02.out.txt}%
\end{example}

첫 번째 예시에서 10일 동안의 스트레스 양은 $[2, 1, 5, 12, 16, 8, 11, 17, 21, 14]$이며, 스트레스의 양이 10 이상인 날은 6일 있다.

두 번째 예시에서 5일 동안의 스트레스 양은 $[1, 0, 0, 0, 1]$이며, 스트레스의 양이 1 이상인 날은 2일 있다.

% \newpage

% \Notes
% 노트, 필요 없으면 주석 처리

\end{problem}